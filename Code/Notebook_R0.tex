% Options for packages loaded elsewhere
\PassOptionsToPackage{unicode}{hyperref}
\PassOptionsToPackage{hyphens}{url}
%
\documentclass[
]{article}
\usepackage{amsmath,amssymb}
\usepackage{iftex}
\ifPDFTeX
  \usepackage[T1]{fontenc}
  \usepackage[utf8]{inputenc}
  \usepackage{textcomp} % provide euro and other symbols
\else % if luatex or xetex
  \usepackage{unicode-math} % this also loads fontspec
  \defaultfontfeatures{Scale=MatchLowercase}
  \defaultfontfeatures[\rmfamily]{Ligatures=TeX,Scale=1}
\fi
\usepackage{lmodern}
\ifPDFTeX\else
  % xetex/luatex font selection
\fi
% Use upquote if available, for straight quotes in verbatim environments
\IfFileExists{upquote.sty}{\usepackage{upquote}}{}
\IfFileExists{microtype.sty}{% use microtype if available
  \usepackage[]{microtype}
  \UseMicrotypeSet[protrusion]{basicmath} % disable protrusion for tt fonts
}{}
\makeatletter
\@ifundefined{KOMAClassName}{% if non-KOMA class
  \IfFileExists{parskip.sty}{%
    \usepackage{parskip}
  }{% else
    \setlength{\parindent}{0pt}
    \setlength{\parskip}{6pt plus 2pt minus 1pt}}
}{% if KOMA class
  \KOMAoptions{parskip=half}}
\makeatother
\usepackage{xcolor}
\usepackage[margin=1in]{geometry}
\usepackage{color}
\usepackage{fancyvrb}
\newcommand{\VerbBar}{|}
\newcommand{\VERB}{\Verb[commandchars=\\\{\}]}
\DefineVerbatimEnvironment{Highlighting}{Verbatim}{commandchars=\\\{\}}
% Add ',fontsize=\small' for more characters per line
\usepackage{framed}
\definecolor{shadecolor}{RGB}{248,248,248}
\newenvironment{Shaded}{\begin{snugshade}}{\end{snugshade}}
\newcommand{\AlertTok}[1]{\textcolor[rgb]{0.94,0.16,0.16}{#1}}
\newcommand{\AnnotationTok}[1]{\textcolor[rgb]{0.56,0.35,0.01}{\textbf{\textit{#1}}}}
\newcommand{\AttributeTok}[1]{\textcolor[rgb]{0.13,0.29,0.53}{#1}}
\newcommand{\BaseNTok}[1]{\textcolor[rgb]{0.00,0.00,0.81}{#1}}
\newcommand{\BuiltInTok}[1]{#1}
\newcommand{\CharTok}[1]{\textcolor[rgb]{0.31,0.60,0.02}{#1}}
\newcommand{\CommentTok}[1]{\textcolor[rgb]{0.56,0.35,0.01}{\textit{#1}}}
\newcommand{\CommentVarTok}[1]{\textcolor[rgb]{0.56,0.35,0.01}{\textbf{\textit{#1}}}}
\newcommand{\ConstantTok}[1]{\textcolor[rgb]{0.56,0.35,0.01}{#1}}
\newcommand{\ControlFlowTok}[1]{\textcolor[rgb]{0.13,0.29,0.53}{\textbf{#1}}}
\newcommand{\DataTypeTok}[1]{\textcolor[rgb]{0.13,0.29,0.53}{#1}}
\newcommand{\DecValTok}[1]{\textcolor[rgb]{0.00,0.00,0.81}{#1}}
\newcommand{\DocumentationTok}[1]{\textcolor[rgb]{0.56,0.35,0.01}{\textbf{\textit{#1}}}}
\newcommand{\ErrorTok}[1]{\textcolor[rgb]{0.64,0.00,0.00}{\textbf{#1}}}
\newcommand{\ExtensionTok}[1]{#1}
\newcommand{\FloatTok}[1]{\textcolor[rgb]{0.00,0.00,0.81}{#1}}
\newcommand{\FunctionTok}[1]{\textcolor[rgb]{0.13,0.29,0.53}{\textbf{#1}}}
\newcommand{\ImportTok}[1]{#1}
\newcommand{\InformationTok}[1]{\textcolor[rgb]{0.56,0.35,0.01}{\textbf{\textit{#1}}}}
\newcommand{\KeywordTok}[1]{\textcolor[rgb]{0.13,0.29,0.53}{\textbf{#1}}}
\newcommand{\NormalTok}[1]{#1}
\newcommand{\OperatorTok}[1]{\textcolor[rgb]{0.81,0.36,0.00}{\textbf{#1}}}
\newcommand{\OtherTok}[1]{\textcolor[rgb]{0.56,0.35,0.01}{#1}}
\newcommand{\PreprocessorTok}[1]{\textcolor[rgb]{0.56,0.35,0.01}{\textit{#1}}}
\newcommand{\RegionMarkerTok}[1]{#1}
\newcommand{\SpecialCharTok}[1]{\textcolor[rgb]{0.81,0.36,0.00}{\textbf{#1}}}
\newcommand{\SpecialStringTok}[1]{\textcolor[rgb]{0.31,0.60,0.02}{#1}}
\newcommand{\StringTok}[1]{\textcolor[rgb]{0.31,0.60,0.02}{#1}}
\newcommand{\VariableTok}[1]{\textcolor[rgb]{0.00,0.00,0.00}{#1}}
\newcommand{\VerbatimStringTok}[1]{\textcolor[rgb]{0.31,0.60,0.02}{#1}}
\newcommand{\WarningTok}[1]{\textcolor[rgb]{0.56,0.35,0.01}{\textbf{\textit{#1}}}}
\usepackage{graphicx}
\makeatletter
\def\maxwidth{\ifdim\Gin@nat@width>\linewidth\linewidth\else\Gin@nat@width\fi}
\def\maxheight{\ifdim\Gin@nat@height>\textheight\textheight\else\Gin@nat@height\fi}
\makeatother
% Scale images if necessary, so that they will not overflow the page
% margins by default, and it is still possible to overwrite the defaults
% using explicit options in \includegraphics[width, height, ...]{}
\setkeys{Gin}{width=\maxwidth,height=\maxheight,keepaspectratio}
% Set default figure placement to htbp
\makeatletter
\def\fps@figure{htbp}
\makeatother
\setlength{\emergencystretch}{3em} % prevent overfull lines
\providecommand{\tightlist}{%
  \setlength{\itemsep}{0pt}\setlength{\parskip}{0pt}}
\setcounter{secnumdepth}{-\maxdimen} % remove section numbering
\ifLuaTeX
  \usepackage{selnolig}  % disable illegal ligatures
\fi
\usepackage{bookmark}
\IfFileExists{xurl.sty}{\usepackage{xurl}}{} % add URL line breaks if available
\urlstyle{same}
\hypersetup{
  pdftitle={Covid Analysis Matteo Bracco},
  hidelinks,
  pdfcreator={LaTeX via pandoc}}

\title{Covid Analysis Matteo Bracco}
\author{}
\date{\vspace{-2.5em}}

\begin{document}
\maketitle

\section{Introduction}\label{introduction}

In this notebook we look for data which can be analyzed through the
Lattice Gas Cellular Automata (LGCA) model, for the model details we
refer to
\href{https://doi.org/10.1016/j.simpat.2008.05.015}{Schneckenreither et
al.}.

The LGCA model addresses the spatial diffusion component in an epidemic
spread, distributing all individuals from the population on a lattice.
Infections are limited to individuals which share the same cell in the
same time steps. A set of partially stochastic rules governs the motion
between cells of the lattice.

To simulate an LGCA model, we will need data coming from a small and
possible sparse population, to be able to work from my Laptop, and to
make the spatial diffusion effect noticeable on the epidemic spread. A
good possibility is to use data coming from low density areas, for
instance some county in Alaska will do. We will firstly try to use those
data to estimate parameters for the SIR model, then we will run a SIR
and LGCA simulation.

We will then use \(\beta\) with the other SIR parameters to run an LGCA
and some Modified LGCA simulations in a
\href{./Covid_Bracco.ipynb}{Jupyther Notebbok}, and we will compare the
results with SIR simulations and real data here on this R Notebook.

Let's start by implementing the necessary library, importing the data
and define some functions to visualize time series with ggplot2. The
data selected comes from the Kodiak Island Area in Alaska. Its
population from 2020 found in \href{https://www.census.gov}{censu.gov}
was of \(13100\) people.

\begin{Shaded}
\begin{Highlighting}[]
\FunctionTok{library}\NormalTok{(}\StringTok{"zoo"}\NormalTok{)                          }\CommentTok{\# For rollmean}
\FunctionTok{library}\NormalTok{(}\StringTok{"ggplot2"}\NormalTok{)                      }\CommentTok{\# For nice time series plots}
\FunctionTok{library}\NormalTok{(deSolve)                        }\CommentTok{\# To solve SIR ODEs}
\FunctionTok{library}\NormalTok{(}\StringTok{"reticulate"}\NormalTok{)                   }\CommentTok{\# To import pyhton files}
\FunctionTok{library}\NormalTok{(tidyr)                          }\CommentTok{\# To plot time series with}
\FunctionTok{library}\NormalTok{(dplyr)                          }\CommentTok{\# ggplot2}
\FunctionTok{library}\NormalTok{(}\StringTok{"gridExtra"}\NormalTok{)                    }\CommentTok{\# to arrange ggplot2 objects into grids}
 

\CommentTok{\# This function allows to plot obs against date, where}
\CommentTok{\# obs is the collection of new registered infections, with respect to the dates}
\CommentTok{\# stored in data}

\CommentTok{\# Time series with only one observation}
\NormalTok{infected\_series}\OtherTok{=}\ControlFlowTok{function}\NormalTok{(date,obs,}\AttributeTok{Title=}\StringTok{"Time Series of Daily New Positives"}\NormalTok{)\{}
\NormalTok{  df}\OtherTok{=}\FunctionTok{data.frame}\NormalTok{(date,obs)}
\NormalTok{  p}\OtherTok{=}\FunctionTok{ggplot}\NormalTok{(df, }\FunctionTok{aes}\NormalTok{(}\AttributeTok{x =} \FunctionTok{as.Date}\NormalTok{(date), }\AttributeTok{y =}\NormalTok{ obs)) }\SpecialCharTok{+}
  \FunctionTok{geom\_line}\NormalTok{()}\SpecialCharTok{+}
  \FunctionTok{xlab}\NormalTok{(}\StringTok{"date"}\NormalTok{)}\SpecialCharTok{+}
    \FunctionTok{ylab}\NormalTok{(}\StringTok{"Infected"}\NormalTok{)}\SpecialCharTok{+}
    \FunctionTok{labs}\NormalTok{(}\AttributeTok{title=}\NormalTok{Title)}
  \FunctionTok{return}\NormalTok{(p)}
\NormalTok{\}}

\CommentTok{\# Time series with multiple observations}
\NormalTok{Multiple\_series}\OtherTok{=}\ControlFlowTok{function}\NormalTok{(df,}\AttributeTok{Title=}\StringTok{"New Infected per day"}\NormalTok{)\{}
\NormalTok{  di }\OtherTok{\textless{}{-}}\NormalTok{ df }\SpecialCharTok{\%\textgreater{}\%}
  \FunctionTok{pivot\_longer}\NormalTok{(}\AttributeTok{cols =} \SpecialCharTok{{-}}\NormalTok{times, }\AttributeTok{names\_to =} \StringTok{"Infected"}\NormalTok{, }\AttributeTok{values\_to =} \StringTok{"Infected\_Obs"}\NormalTok{)}

\NormalTok{  p}\OtherTok{=}\FunctionTok{ggplot}\NormalTok{(di, }\FunctionTok{aes}\NormalTok{(}\AttributeTok{x =} \FunctionTok{as.Date}\NormalTok{(times), }\AttributeTok{y =}\NormalTok{ Infected\_Obs, }\AttributeTok{color =}\NormalTok{ Infected)) }\SpecialCharTok{+}
  \FunctionTok{geom\_line}\NormalTok{(}\AttributeTok{size =} \DecValTok{1}\NormalTok{)}\SpecialCharTok{+}
  \FunctionTok{labs}\NormalTok{(}\AttributeTok{title =}\NormalTok{ Title,}
       \AttributeTok{x =} \StringTok{"Date"}\NormalTok{, }\AttributeTok{y =} \StringTok{"Value"}\NormalTok{, }\AttributeTok{color =} \StringTok{"Observation"}\NormalTok{) }\SpecialCharTok{+}
  \FunctionTok{theme\_minimal}\NormalTok{()}
  \FunctionTok{return}\NormalTok{(p)}
\NormalTok{\}}
\end{Highlighting}
\end{Shaded}

Here is the database data for Covid-19 and the Kodiak Island selection.

\begin{Shaded}
\begin{Highlighting}[]
\NormalTok{infections}\OtherTok{=}\FunctionTok{read.csv}\NormalTok{(}\StringTok{"epidemiology.csv"}\NormalTok{) }\CommentTok{\# Covid Data}
\NormalTok{Inf\_AL}\OtherTok{=}\FunctionTok{subset}\NormalTok{(infections, location\_key }\SpecialCharTok{==} \StringTok{"US\_AK\_02150"}\NormalTok{)}
\end{Highlighting}
\end{Shaded}

\section{Estimating Parameters for our
simulations}\label{estimating-parameters-for-our-simulations}

\subsection{\texorpdfstring{Estimating
\(\beta\)}{Estimating \textbackslash beta}}\label{estimating-beta}

We start by plotting the registered infected population over time.

\begin{Shaded}
\begin{Highlighting}[]
\FunctionTok{infected\_series}\NormalTok{(Inf\_AL}\SpecialCharTok{$}\NormalTok{date,Inf\_AL}\SpecialCharTok{$}\NormalTok{new\_confirmed)}
\end{Highlighting}
\end{Shaded}

\includegraphics{Notebook_R0_files/figure-latex/unnamed-chunk-3-1.pdf}
We can see that there are three separate epidemic peeks. Note that for
the last two peeks, there is a big heterogeneity in the registration of
new infected, represented by the big jumps in our time series, this can
be caused by different flows in the data collection procedure, for
instance all the new cases where registered at the end of the week
instead of being registered day by day.

This pehonomenon is also present in the first peak, but it's less
impactful, which probably means that the lower number of registered
people can be linked to the lower number of tested people, which
unfortunately is not an available data for this dataset.

We will then try to work on the first peak, as zero values would be an
issue when estimating \(\beta\).

For starting, let's visualize the infected registrations from the
\(10^{th}\) of November 2020 to the \(31^{st}\) of December 2020.

\begin{Shaded}
\begin{Highlighting}[]
\NormalTok{start}\OtherTok{=}\FunctionTok{which}\NormalTok{(Inf\_AL}\SpecialCharTok{$}\NormalTok{date}\SpecialCharTok{==}\StringTok{"2020{-}10{-}01"}\NormalTok{)}
\NormalTok{stop}\OtherTok{=}\FunctionTok{which}\NormalTok{(Inf\_AL}\SpecialCharTok{$}\NormalTok{date}\SpecialCharTok{==}\StringTok{"2020{-}12{-}31"}\NormalTok{)}
\NormalTok{obs}\OtherTok{=}\NormalTok{Inf\_AL}\SpecialCharTok{$}\NormalTok{new\_confirmed[start}\SpecialCharTok{:}\NormalTok{stop]}
\NormalTok{date}\OtherTok{=}\NormalTok{Inf\_AL}\SpecialCharTok{$}\NormalTok{date[start}\SpecialCharTok{:}\NormalTok{stop]}
\FunctionTok{infected\_series}\NormalTok{(date,obs,}\StringTok{"New Positives October{-}December 2020"}\NormalTok{)}
\end{Highlighting}
\end{Shaded}

\includegraphics{Notebook_R0_files/figure-latex/unnamed-chunk-4-1.pdf}

We choose a reasonable time interval as the exponential growth period of
the new confirmed cases:

\begin{Shaded}
\begin{Highlighting}[]
\NormalTok{start2}\OtherTok{=}\FunctionTok{which}\NormalTok{(date}\SpecialCharTok{==}\StringTok{"2020{-}11{-}27"}\NormalTok{)}
\NormalTok{stop2}\OtherTok{=}\FunctionTok{which}\NormalTok{(date}\SpecialCharTok{==}\StringTok{"2020{-}12{-}10"}\NormalTok{)}
\NormalTok{date\_exp}\OtherTok{=}\NormalTok{date[start2}\SpecialCharTok{:}\NormalTok{stop2]}
\NormalTok{obs\_exp}\OtherTok{=}\NormalTok{obs[start2}\SpecialCharTok{:}\NormalTok{stop2]}

\FunctionTok{infected\_series}\NormalTok{(date\_exp,obs\_exp, }\StringTok{"Exponential Growth in New Positives"}\NormalTok{ )}
\end{Highlighting}
\end{Shaded}

\includegraphics{Notebook_R0_files/figure-latex/unnamed-chunk-5-1.pdf}
We can then fit this data to estimate \(\beta\), see
\href{https://doi.org/10.1016/j.idm.2019.12.009}{Junling Ma}. To
understand how we will estimate \(\beta\) let's refresh the SIR
equations. Let \(N\) be the population size, then
\[ \frac{dS}{dt} = -\frac \beta N SI\]
\[ \frac{dI}{dt} = \frac \beta N SI-\gamma I \]
\[ \frac{dR}{dt} = \gamma I \] In the beginning of the epidemic curve,
we can suppose \(S \simeq N\). Let \(C=\frac \beta N SI\) be the
incidence number, i.e the new cases per day, then we have that
\(C\simeq \beta I=\beta I_0e^{\lambda \cdot t}\), where
\(\lambda= \beta-\gamma\). We can fit the new confirmed cases in the
exponential growth period to estimate \(\beta\). Note that, even if we
only register a fraction \(p\in (0,1)\) of the real new positive cases,
the fitting process will not change the \(\beta\) estimation:

\[pC\simeq p\beta I_0e^{\lambda \cdot t}= C_0 e^{\lambda \cdot t}\] If
\(pC\) is growing exponential, we can fit the log of our data which will
be growing linearly, precisely \[log(pC)\simeq K_0 +\lambda t\] where
\(K_0=log(C_0)\).

\begin{Shaded}
\begin{Highlighting}[]
\NormalTok{n}\OtherTok{=}\FunctionTok{length}\NormalTok{(date\_exp)}
\NormalTok{t}\OtherTok{=}\FunctionTok{c}\NormalTok{(}\DecValTok{1}\SpecialCharTok{:}\NormalTok{n)}
\NormalTok{log\_obs}\OtherTok{=}\FunctionTok{log}\NormalTok{(obs\_exp)}
\NormalTok{Fit}\OtherTok{=} \FunctionTok{lm}\NormalTok{(log\_obs }\SpecialCharTok{\textasciitilde{}}\NormalTok{ t)}
\NormalTok{lambda }\OtherTok{\textless{}{-}} \FunctionTok{coef}\NormalTok{(Fit)[[}\DecValTok{2}\NormalTok{]]}
\NormalTok{K0}\OtherTok{=}\FunctionTok{coef}\NormalTok{(Fit)[[}\DecValTok{1}\NormalTok{]]}
\NormalTok{gamma}\OtherTok{=}\DecValTok{1}\SpecialCharTok{/}\FloatTok{6.5}
\NormalTok{beta}\OtherTok{=}\NormalTok{lambda}\SpecialCharTok{+}\NormalTok{gamma}
\FunctionTok{summary}\NormalTok{(Fit)}
\end{Highlighting}
\end{Shaded}

\begin{verbatim}
## 
## Call:
## lm(formula = log_obs ~ t)
## 
## Residuals:
##      Min       1Q   Median       3Q      Max 
## -1.62578 -0.31753  0.05835  0.44151  1.29955 
## 
## Coefficients:
##             Estimate Std. Error t value Pr(>|t|)   
## (Intercept)   1.4337     0.3985   3.598  0.00366 **
## t             0.1920     0.0468   4.103  0.00146 **
## ---
## Signif. codes:  0 '***' 0.001 '**' 0.01 '*' 0.05 '.' 0.1 ' ' 1
## 
## Residual standard error: 0.706 on 12 degrees of freedom
## Multiple R-squared:  0.5838, Adjusted R-squared:  0.5491 
## F-statistic: 16.83 on 1 and 12 DF,  p-value: 0.001465
\end{verbatim}

\begin{Shaded}
\begin{Highlighting}[]
\NormalTok{df\_fit }\OtherTok{\textless{}{-}} \FunctionTok{data.frame}\NormalTok{(}\AttributeTok{t =}\NormalTok{ t, }\AttributeTok{new\_confirmed =}\NormalTok{ obs\_exp)}

\NormalTok{C0}\OtherTok{=}\FunctionTok{exp}\NormalTok{(K0)}
\CommentTok{\# Add fitted values from the model}
\NormalTok{df\_fit}\SpecialCharTok{$}\NormalTok{fit }\OtherTok{\textless{}{-}} \FunctionTok{exp}\NormalTok{(}\FunctionTok{predict}\NormalTok{(Fit))}
\CommentTok{\# Plot log{-}transformed data and linear fit}
\FunctionTok{ggplot}\NormalTok{(df\_fit, }\FunctionTok{aes}\NormalTok{(}\AttributeTok{x =} \FunctionTok{as.Date}\NormalTok{(date\_exp))) }\SpecialCharTok{+}
  \FunctionTok{geom\_line}\NormalTok{(}\FunctionTok{aes}\NormalTok{(}\AttributeTok{y =}\NormalTok{ obs\_exp, }\AttributeTok{color =} \StringTok{"Real Data"}\NormalTok{), }\AttributeTok{size =} \FloatTok{0.1}\NormalTok{) }\SpecialCharTok{+}  \CommentTok{\# Real Data line}
  \FunctionTok{geom\_point}\NormalTok{(}\FunctionTok{aes}\NormalTok{(}\AttributeTok{y =}\NormalTok{ obs\_exp, }\AttributeTok{color =} \StringTok{"Real Data"}\NormalTok{), }\AttributeTok{size =} \DecValTok{3}\NormalTok{)}\SpecialCharTok{+}
  \FunctionTok{geom\_line}\NormalTok{(}\FunctionTok{aes}\NormalTok{(}\AttributeTok{y =}\NormalTok{ fit, }\AttributeTok{color =} \StringTok{"Fitted Model"}\NormalTok{), }\AttributeTok{size =} \DecValTok{1}\NormalTok{) }\SpecialCharTok{+}      \CommentTok{\# Fitted Model line}
  \FunctionTok{labs}\NormalTok{(}\AttributeTok{title =} \StringTok{"Daily New Infections with Exponential Fit"}\NormalTok{,}
       \AttributeTok{x =} \StringTok{"Date"}\NormalTok{,}
       \AttributeTok{y =} \StringTok{"log(I\_new)"}\NormalTok{) }\SpecialCharTok{+}
  \FunctionTok{scale\_color\_manual}\NormalTok{(}\AttributeTok{name=}\StringTok{\textquotesingle{}Fitting New Confirmed Cases\textquotesingle{}}\NormalTok{,}
                     \AttributeTok{breaks=}\FunctionTok{c}\NormalTok{(}\StringTok{\textquotesingle{}Real Data\textquotesingle{}}\NormalTok{, }\StringTok{\textquotesingle{}Fitted Model\textquotesingle{}}\NormalTok{),}
                     \AttributeTok{values=}\FunctionTok{c}\NormalTok{(}\StringTok{\textquotesingle{}Real Data\textquotesingle{}}\OtherTok{=}\StringTok{\textquotesingle{}black\textquotesingle{}}\NormalTok{, }\StringTok{\textquotesingle{}Fitted Model\textquotesingle{}}\OtherTok{=}\StringTok{\textquotesingle{}tomato\textquotesingle{}}\NormalTok{))}\SpecialCharTok{+}
  \FunctionTok{theme}\NormalTok{(}\AttributeTok{legend.position=}\FunctionTok{c}\NormalTok{(}\FloatTok{0.2}\NormalTok{,}\FloatTok{0.8}\NormalTok{))}
\end{Highlighting}
\end{Shaded}

\includegraphics{Notebook_R0_files/figure-latex/unnamed-chunk-7-1.pdf}

This is the value of \(\beta\) we will use in the simulations

\subsection{The Initial Number of Infected People and the Starting
Date}\label{the-initial-number-of-infected-people-and-the-starting-date}

We now need to understand when the epidemic started and the initial
number of infected people. To do so we can visualize the number of new
infected in the months priors to November

\begin{Shaded}
\begin{Highlighting}[]
\NormalTok{start}\OtherTok{=}\FunctionTok{which}\NormalTok{(Inf\_AL}\SpecialCharTok{$}\NormalTok{date}\SpecialCharTok{==}\StringTok{"2020{-}07{-}01"}\NormalTok{)}
\NormalTok{stop}\OtherTok{=}\FunctionTok{which}\NormalTok{(Inf\_AL}\SpecialCharTok{$}\NormalTok{date}\SpecialCharTok{==}\StringTok{"2020{-}10{-}31"}\NormalTok{)}
\NormalTok{obs}\OtherTok{=}\NormalTok{Inf\_AL}\SpecialCharTok{$}\NormalTok{new\_confirmed[start}\SpecialCharTok{:}\NormalTok{stop]}
\NormalTok{date}\OtherTok{=}\NormalTok{Inf\_AL}\SpecialCharTok{$}\NormalTok{date[start}\SpecialCharTok{:}\NormalTok{stop]}
\FunctionTok{infected\_series}\NormalTok{(date,obs)}
\end{Highlighting}
\end{Shaded}

\includegraphics{Notebook_R0_files/figure-latex/unnamed-chunk-8-1.pdf}
As we can see there was a small peak around August, afterwards the
situation seemed to be stable with almost no new positive cases between
September and the start of October Hence we can reasonably start our
Simulation at the beginning of October. Note that previously there where
not some big wawes of Covid-19 in the Area, so we can assume the initial
value of recovered people to be \(0\) as well.

The other option would be to simulate the epidemic from the start of
July, however the \(\beta\) estimation ha a local character. Hence it is
best to limit our simulations to a span of \(100\) days across the
autumn/winter season in Alaska, which also keeps reasonable the
computational time for the Simulations.

The number of infected people at October \(1^{st}\) is the most
difficult parameter to estimate, we can start by evaluating the sum in
the \(5\) preceding days of the registered infected individuals

\begin{Shaded}
\begin{Highlighting}[]
\NormalTok{end}\OtherTok{=}\FunctionTok{which}\NormalTok{(Inf\_AL}\SpecialCharTok{$}\NormalTok{date}\SpecialCharTok{==}\StringTok{"2020{-}10{-}01"}\NormalTok{)}
\NormalTok{start}\OtherTok{=}\NormalTok{end}\DecValTok{{-}5}
\FunctionTok{sum}\NormalTok{(Inf\_AL}\SpecialCharTok{$}\NormalTok{new\_confirmed[start}\SpecialCharTok{:}\NormalTok{end])}
\end{Highlighting}
\end{Shaded}

\begin{verbatim}
## [1] 5
\end{verbatim}

This estimate is a probably a lower bound on the real number of infected
people for October \$1\^{}\{st\}. We can at least triple such number to
represent the initial population of infected people. However if the
models are robust enough, we shall not expect small changes in the
initial values to compromise the quality development of our simulations,
which is the aspect which interest us the most.

\subsection{Parameters for the
Simulations}\label{parameters-for-the-simulations}

We can the visualize the initial values and parameters for our LGCA and
SIR simulations below.

\begin{Shaded}
\begin{Highlighting}[]
\NormalTok{startN}\OtherTok{=}\FunctionTok{which}\NormalTok{(Inf\_AL}\SpecialCharTok{$}\NormalTok{date}\SpecialCharTok{==}\StringTok{"2020{-}10{-}01"}\NormalTok{)}
\NormalTok{endN}\OtherTok{=}\FunctionTok{which}\NormalTok{(Inf\_AL}\SpecialCharTok{$}\NormalTok{date}\SpecialCharTok{==}\StringTok{"2021{-}01{-}09"}\NormalTok{)}
\NormalTok{I0}\OtherTok{=}\DecValTok{15}
\NormalTok{date0}\OtherTok{=}\StringTok{"2020{-}10{-}01"}
\NormalTok{N}\OtherTok{=}\DecValTok{13100}
\NormalTok{gamma}\OtherTok{=}\DecValTok{1}\SpecialCharTok{/}\FloatTok{6.5}

\NormalTok{Parameter}\OtherTok{=}\FunctionTok{c}\NormalTok{(}\StringTok{"Starting Date"}\NormalTok{,}\StringTok{"Total Population"}\NormalTok{,}\StringTok{"Initial Population of Infected People"}\NormalTok{,}\StringTok{"gamma"}\NormalTok{,}\StringTok{"beta"}\NormalTok{)}
\NormalTok{Value}\OtherTok{=}\FunctionTok{c}\NormalTok{(date0,N,I0,gamma,beta)}
\NormalTok{Initial}\OtherTok{=}\FunctionTok{data.frame}\NormalTok{(Parameter,Value)}
\NormalTok{Initial}
\end{Highlighting}
\end{Shaded}

\begin{verbatim}
##                               Parameter             Value
## 1                         Starting Date        2020-10-01
## 2                      Total Population             13100
## 3 Initial Population of Infected People                15
## 4                                 gamma 0.153846153846154
## 5                                  beta  0.34588132363107
\end{verbatim}

\section{Simulations Comparisons}\label{simulations-comparisons}

We can load the number of new infected per day simulated via the LGCA
simulation in python. The idea is to compare such results with real data
and results of the classic SIR model.

\begin{Shaded}
\begin{Highlighting}[]
\NormalTok{history}\OtherTok{=}\FunctionTok{py\_load\_object}\NormalTok{(}\StringTok{"LGCAv4.pckl"}\NormalTok{)}
\end{Highlighting}
\end{Shaded}

\subsection{SIR Model}\label{sir-model}

We then need to evaluate a classic version of the SIR model, we will use
the simplest version as it is the one on which the LGCA model is built
on.

\begin{Shaded}
\begin{Highlighting}[]
\NormalTok{sir\_equations }\OtherTok{\textless{}{-}} \ControlFlowTok{function}\NormalTok{(time, variables, parameters) \{ }\CommentTok{\# SIR equation}
  \FunctionTok{with}\NormalTok{(}\FunctionTok{as.list}\NormalTok{(}\FunctionTok{c}\NormalTok{(variables, parameters)), \{}
\NormalTok{    dS }\OtherTok{\textless{}{-}} \SpecialCharTok{{-}}\NormalTok{B }\SpecialCharTok{*}\NormalTok{ I }\SpecialCharTok{*}\NormalTok{ S}
\NormalTok{    dI }\OtherTok{\textless{}{-}}\NormalTok{  B }\SpecialCharTok{*}\NormalTok{ I }\SpecialCharTok{*}\NormalTok{ S }\SpecialCharTok{{-}}\NormalTok{ G }\SpecialCharTok{*}\NormalTok{ I}
\NormalTok{    dR }\OtherTok{\textless{}{-}}\NormalTok{  G }\SpecialCharTok{*}\NormalTok{ I}
    \FunctionTok{return}\NormalTok{(}\FunctionTok{list}\NormalTok{(}\FunctionTok{c}\NormalTok{(dS, dI, dR)))}
\NormalTok{  \})}
\NormalTok{\}}

\NormalTok{parameters\_values }\OtherTok{\textless{}{-}} \FunctionTok{c}\NormalTok{( }\CommentTok{\# Params for SIR}
  \AttributeTok{B  =}\NormalTok{ beta}\SpecialCharTok{/}\NormalTok{N, }
  \AttributeTok{G =}\NormalTok{ gamma}
\NormalTok{)}

\NormalTok{initial\_values }\OtherTok{\textless{}{-}} \FunctionTok{c}\NormalTok{( }\CommentTok{\# Intial Values}
  \AttributeTok{S =}\NormalTok{ N}\SpecialCharTok{{-}}\NormalTok{I0,  }
  \AttributeTok{I =}\NormalTok{   I0,}
  \AttributeTok{R =}   \DecValTok{0}  
\NormalTok{)}

\NormalTok{time\_values }\OtherTok{\textless{}{-}} \FunctionTok{seq}\NormalTok{(}\DecValTok{0}\NormalTok{, }\DecValTok{100}\NormalTok{) }\CommentTok{\# 100 days}

\NormalTok{SIR\_obs }\OtherTok{=}\FunctionTok{as.data.frame}\NormalTok{(}\FunctionTok{ode}\NormalTok{( }\CommentTok{\#Solvin the ODE}
  \AttributeTok{y =}\NormalTok{ initial\_values,}
  \AttributeTok{times =}\NormalTok{ time\_values,}
  \AttributeTok{func =}\NormalTok{ sir\_equations,}
  \AttributeTok{parms =}\NormalTok{ parameters\_values }
\NormalTok{))}
\end{Highlighting}
\end{Shaded}

We can derive the incidence of the susceptibles with the following
functions, an analogous version can be found in the Python script

\begin{Shaded}
\begin{Highlighting}[]
\NormalTok{find\_incid }\OtherTok{\textless{}{-}} \ControlFlowTok{function}\NormalTok{(I,R)\{}
\NormalTok{  n }\OtherTok{\textless{}{-}} \FunctionTok{length}\NormalTok{(I)}
\NormalTok{  incid }\OtherTok{\textless{}{-}} \FunctionTok{c}\NormalTok{(}\DecValTok{15}\NormalTok{)}
  \ControlFlowTok{for}\NormalTok{ (i }\ControlFlowTok{in} \DecValTok{2}\SpecialCharTok{:}\NormalTok{n) \{}
\NormalTok{    I\_today }\OtherTok{\textless{}{-}}\NormalTok{ I[i]}
\NormalTok{    R\_new }\OtherTok{\textless{}{-}}\NormalTok{ R[i] }\SpecialCharTok{{-}}\NormalTok{ R[i}\DecValTok{{-}1}\NormalTok{]}
\NormalTok{    I\_yest }\OtherTok{\textless{}{-}}\NormalTok{ I[i}\DecValTok{{-}1}\NormalTok{] }\SpecialCharTok{{-}}\NormalTok{ R\_new}
\NormalTok{    I\_new }\OtherTok{\textless{}{-}}\NormalTok{ I\_today }\SpecialCharTok{{-}}\NormalTok{ I\_yest}
\NormalTok{    incid }\OtherTok{\textless{}{-}} \FunctionTok{c}\NormalTok{(incid, I\_new)}
\NormalTok{  \}}
  
  \FunctionTok{return}\NormalTok{(incid)}
\NormalTok{\}}

\NormalTok{incid\_SIR}\OtherTok{=}\FunctionTok{find\_incid}\NormalTok{(SIR\_obs}\SpecialCharTok{$}\NormalTok{I,SIR\_obs}\SpecialCharTok{$}\NormalTok{R)}
\end{Highlighting}
\end{Shaded}

We can then build a data.frame containing real data, LGCA data and SIR
data and compare the results

\begin{Shaded}
\begin{Highlighting}[]
\NormalTok{start}\OtherTok{=}\FunctionTok{which}\NormalTok{(Inf\_AL}\SpecialCharTok{$}\NormalTok{date}\SpecialCharTok{==}\NormalTok{date0)}
\NormalTok{stop}\OtherTok{=}\NormalTok{start}\SpecialCharTok{+}\DecValTok{100}
\NormalTok{times}\OtherTok{=}\NormalTok{Inf\_AL}\SpecialCharTok{$}\NormalTok{date[start}\SpecialCharTok{:}\NormalTok{stop]}
\NormalTok{incid\_real}\OtherTok{=}\NormalTok{Inf\_AL}\SpecialCharTok{$}\NormalTok{new\_confirmed[start}\SpecialCharTok{:}\NormalTok{stop]}
\NormalTok{Data\_Incid}\OtherTok{=}\FunctionTok{data.frame}\NormalTok{(times,incid\_SIR,history,incid\_real)}
\FunctionTok{colnames}\NormalTok{(Data\_Incid) }\OtherTok{\textless{}{-}} \FunctionTok{c}\NormalTok{(}\StringTok{"times"}\NormalTok{,}\StringTok{"SIR"}\NormalTok{,}\StringTok{"LGCA"}\NormalTok{,}\StringTok{"Real Data"}\NormalTok{)}
\FunctionTok{Multiple\_series}\NormalTok{(Data\_Incid,}\StringTok{"SIR and LGCA Simulations Compared with Real Data"}\NormalTok{)}
\end{Highlighting}
\end{Shaded}

\includegraphics{Notebook_R0_files/figure-latex/unnamed-chunk-14-1.pdf}

The main issue we are trying to address with the LGCA model is quite
noticeable from this simulation: the real peak of new infected people is
shifted in time with respect to the SIR estimates.

Now, we would like to see the peak of the LGCA estimate near to the peak
of the real data, however we do not even see a peak for the LGCA
simulation from this visualization. Let's remove the SIR data to have a
better view of the LGCA results

\begin{Shaded}
\begin{Highlighting}[]
\NormalTok{Incid\_new}\OtherTok{=}\FunctionTok{data.frame}\NormalTok{(times,history,incid\_real)}
\FunctionTok{colnames}\NormalTok{(Incid\_new) }\OtherTok{\textless{}{-}} \FunctionTok{c}\NormalTok{(}\StringTok{"times"}\NormalTok{,}\StringTok{"LGCA"}\NormalTok{,}\StringTok{"Real Data"}\NormalTok{)}
\FunctionTok{Multiple\_series}\NormalTok{(Incid\_new, }\StringTok{"LGCA vs Reald Data"}\NormalTok{)}
\end{Highlighting}
\end{Shaded}

\includegraphics{Notebook_R0_files/figure-latex/unnamed-chunk-15-1.pdf}
The LGCA model seems in fact to have a very small and very large peak
\(20\) days after the SIR model, quite closer to the real registered
peak, but the estimates of the new positive cases are very low, even
when compared with the registered data, and the peak is barely
noticeable, and we can't really exclude that it only happened because of
stochastic effects of this specific simulation. Obviously the best
solution would be to run more simulations and take the average to solve
such issue, but we do not have the computational power to do so.

Nevertheless, the development of the LGCA simulation doesn't seem to
follow the real epidemic spread, even from a qualitative point of view,
as we should at least see a significant increase in the new registered
cases.

This is probably due to the restrictive law of motions and the large
grid imposed by the LGCA model.

However, such restrictions where not justified theoretically in
\href{https://doi.org/10.1016/j.simpat.2008.05.015}{Schneckenreither et
al.}., hence we may as well change them. The Modified LGCA algorithms
removes the restriction imposed by the law of motions between neighbor
cells, as well as removing the ``PacMan Effect''.

Moreover we allow for smaller squared grids which also drastically
decrease the computational time for each simulation. We can for example
compare the effects of this Modified LGCA algorithm on squared grid of
dimension \(10k,\quad k=1,\dots,10\).

\begin{Shaded}
\begin{Highlighting}[]
\NormalTok{new\_history}\OtherTok{=}\FunctionTok{py\_load\_object}\NormalTok{(}\StringTok{"modified10v4.pckl"}\NormalTok{)}
\NormalTok{N}\OtherTok{=}\FunctionTok{length}\NormalTok{(incid\_SIR)}

\NormalTok{pad }\OtherTok{\textless{}{-}} \ControlFlowTok{function}\NormalTok{(x, len) \{}
  \FunctionTok{c}\NormalTok{(x, }\FunctionTok{rep}\NormalTok{(}\DecValTok{0}\NormalTok{, len }\SpecialCharTok{{-}} \FunctionTok{length}\NormalTok{(x)))}
\NormalTok{\}}


\NormalTok{n}\OtherTok{=}\DecValTok{10}
\NormalTok{plots}\OtherTok{=}\FunctionTok{c}\NormalTok{()}
\NormalTok{i}\OtherTok{=}\DecValTok{0}
\ControlFlowTok{for}\NormalTok{(simul }\ControlFlowTok{in}\NormalTok{ new\_history)\{}
\NormalTok{  i}\OtherTok{=}\NormalTok{i}\SpecialCharTok{+}\DecValTok{1}
\NormalTok{  third\_lab}\OtherTok{=}\FunctionTok{paste}\NormalTok{(}\StringTok{"n="}\NormalTok{,n)}
\NormalTok{  pad\_sim}\OtherTok{=}\FunctionTok{pad}\NormalTok{(simul,N)}
\NormalTok{  Modified\_data}\OtherTok{=}\FunctionTok{data.frame}\NormalTok{(times,incid\_real,pad\_sim,incid\_SIR)}
  \FunctionTok{colnames}\NormalTok{(Modified\_data) }\OtherTok{\textless{}{-}} \FunctionTok{c}\NormalTok{(}\StringTok{"times"}\NormalTok{,}\StringTok{"Real Data"}\NormalTok{,  third\_lab,}\StringTok{"SIR"}\NormalTok{)}
\NormalTok{  p}\OtherTok{=}\FunctionTok{Multiple\_series}\NormalTok{(Modified\_data)}
\NormalTok{  n}\OtherTok{=}\NormalTok{n}\SpecialCharTok{+}\DecValTok{10}
\NormalTok{  plots[[i]]}\OtherTok{=}\NormalTok{p}
\NormalTok{\}}



\FunctionTok{grid.arrange}\NormalTok{(}\AttributeTok{grobs =}\NormalTok{ plots, }\AttributeTok{ncol =} \DecValTok{3}\NormalTok{)}
\end{Highlighting}
\end{Shaded}

\includegraphics{Notebook_R0_files/figure-latex/unnamed-chunk-16-1.pdf}

\begin{Shaded}
\begin{Highlighting}[]
\NormalTok{n}\OtherTok{=}\DecValTok{10}
\NormalTok{plots}\OtherTok{=}\FunctionTok{c}\NormalTok{()}
\NormalTok{i}\OtherTok{=}\DecValTok{0}
\ControlFlowTok{for}\NormalTok{(simul }\ControlFlowTok{in}\NormalTok{ new\_history)\{}
\NormalTok{  i}\OtherTok{=}\NormalTok{i}\SpecialCharTok{+}\DecValTok{1}
\NormalTok{  third\_lab}\OtherTok{=}\FunctionTok{paste}\NormalTok{(}\StringTok{"n="}\NormalTok{,n)}
\NormalTok{  pad\_sim}\OtherTok{=}\FunctionTok{pad}\NormalTok{(simul,N)}
\NormalTok{  Modified\_data}\OtherTok{=}\FunctionTok{data.frame}\NormalTok{(times,pad\_sim,incid\_real)}
  \FunctionTok{colnames}\NormalTok{(Modified\_data) }\OtherTok{\textless{}{-}} \FunctionTok{c}\NormalTok{(}\StringTok{"times"}\NormalTok{,  third\_lab,}\StringTok{"Real Data"}\NormalTok{)}
\NormalTok{  p}\OtherTok{=}\FunctionTok{Multiple\_series}\NormalTok{(Modified\_data)}
\NormalTok{  n}\OtherTok{=}\NormalTok{n}\SpecialCharTok{+}\DecValTok{10}
\NormalTok{  plots[[i]]}\OtherTok{=}\NormalTok{p}
\NormalTok{\}}

\FunctionTok{grid.arrange}\NormalTok{(}\AttributeTok{grobs =}\NormalTok{ plots, }\AttributeTok{ncol =} \DecValTok{3}\NormalTok{)}
\end{Highlighting}
\end{Shaded}

\includegraphics{Notebook_R0_files/figure-latex/unnamed-chunk-17-1.pdf}

The modified LGCA simulations presents some interesting features linked
to the base dimension \(d\) of the lattice,

When \(d=10\) the simulation seems to behave accordingly to the SIR
simulation. When we start increasing the lattice dimension, we see a
shift in the peak of the simulation, both down and right. This is
probably due to the spatial diffusion becoming a factor: individuals in
cells further a part from initial outbreaks are not subjected to the
virus in the initial stage of the epidemic.

The peak seems to match the real data peak around \(d=80,90\), but we
should take this information with caution, as we can't really be sure on
the starting day of the epidemic from real data.

However, there is certainly a delay between the peaks of SIR and real
data, which seems to be at least partially correctable through the
Modified Lattice Model.

Note that different laws of motions, and different initial distributions
of the population on the lattice may results in different results in the
qualitative behaviour of the Modified LGCA simulations.

Once \(d\) becomes too large, the peak seems to flatten out and the
trend of new positives seems to be rather stable in time, similarly to
what we saw in the original LGCA model. We do not know if the peak will
arrive later on in time, but it seems quite unlikely. This suggests that
the biggest difference between the two models resides in the grid
dimensions rather then the different laws of motions.

In conclusion, the results of the Modified LGCA simulation seems to
suggest that incorporating correctly the spatial diffusion component
through a lattice representation may help to explain the qualitative
behaviour of an epidemic. Moreover, these results leave open different
questions on the dynamical system described by the LGCA modification
algorithm, which to me seems important to better understand how spatial
diffusion can be modeled in epidemic spreads:

\begin{itemize}
\item
  Can we define an optimal value for the dimension \(d\) of the grid in
  the Modified LGCA simulation? Should it depend on the individuals
  habits, the population density and the geographical structure of the
  area we are studying, and the other parameters (\(\beta,\gamma\)) of
  our model?
\item
  Is the dynamical systems sensitive to the initial value of positive
  people \(I_0\)? At least for small changes in small values of \(I_0\),
  is the system stable? Which are the equilibrium points of the system?
  Are there periodic solutions which may arise in real scenarios?
\item
  How does the peak of infected people relates to the lattice dimension?
  Does the peak flattens out for \(d \simeq \sqrt{N}\) or does it appear
  later on in time? Is it possible to find a formula to link the peak of
  infected people, for instance \(t_{max}=argmax_{t \in T} C(t)\) to the
  model parameters, especially to the grid dimension \(d\)?
\item
  Can we introduce different law of motions to incorporate lockdowns or
  restriction to movement? What happens if we change the initial
  distribution of Infected people on the lattice? Could we use different
  representation for the lattice (e.g rectangular), to better model the
  geography of the area we are analyzing?
\end{itemize}

To end the notebook, let's run \(30\) simulations with dimension
\(d=80\), to balance out the stochastic effects. We can compare the
results with real data, keep always in mind that we are not sure about
the starting date and the initial conditions of the epidemic

\begin{Shaded}
\begin{Highlighting}[]
\NormalTok{v40}\OtherTok{=}\FunctionTok{py\_load\_object}\NormalTok{(}\StringTok{"v40.pckl"}\NormalTok{)}
\NormalTok{v40\_df}\OtherTok{=}\FunctionTok{as.data.frame}\NormalTok{(v40)}
\NormalTok{means40}\OtherTok{=}\FunctionTok{rowMeans}\NormalTok{(v40\_df)}
\NormalTok{Mod40}\OtherTok{=}\FunctionTok{data.frame}\NormalTok{(times,incid\_real,means40)}
\FunctionTok{colnames}\NormalTok{(Mod40)}\OtherTok{=}\FunctionTok{c}\NormalTok{(}\StringTok{"times"}\NormalTok{,}\StringTok{"Real Data"}\NormalTok{,}\StringTok{"Modified LGCA with d=80"}\NormalTok{)}
\FunctionTok{Multiple\_series}\NormalTok{(Mod40,}\StringTok{"Modified LGCA averaged over 30 simulations (d=80) vs Real Data"}\NormalTok{)}
\end{Highlighting}
\end{Shaded}

\includegraphics{Notebook_R0_files/figure-latex/unnamed-chunk-18-1.pdf}

\section{\texorpdfstring{\textbf{References}}{References}}\label{references}

\begin{itemize}
\item
  Schneckenreither, Gunter and Popper, Nikolas and Zauner, Gunther and
  Breitenecker, Felix. \emph{Modelling SIR-type epidemics by ODEs, PDEs,
  difference equations and cellular automata--A comparative study}.
  Simulation Modelling Practice and Theory 16(8)1014-1023 2008.
  \url{https://doi.org/10.1016/j.simpat.2008.05.015}
\item
  Junling Ma. \emph{Estimating epidemic exponential growth rate and
  basic reproduction number} Infectious Disease Modelling 5:129-141
  2020. \url{https://doi.org/10.1016/j.idm.2019.12.009}
\end{itemize}

\end{document}
